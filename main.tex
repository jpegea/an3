\documentclass[12pt]{book}

\usepackage{mathtools, amssymb, amsthm}
\usepackage[catalan]{babel}
\usepackage[papersize={150mm,200mm}, margin=0.5cm, includeheadfoot]{geometry}
\usepackage{graphicx}
\usepackage[dvipsnames]{xcolor}
\usepackage[vvarbb, helvratio=.87, nott]{newtx}
\usepackage[bf, sf]{titlesec}
\usepackage[shortlabels]{enumitem}
\usepackage{hyperref}
\usepackage{tikz}
\usepackage{pgfplots}

\author{Quim Pascual Egea}
\title{Anàlisi Matemàtica III}
\date{2025}

\hypersetup{
  colorlinks=true,
  linkcolor=Violet}

\pgfplotsset{width=5cm,compat=1.9}
\usepgfplotslibrary{external}
\tikzexternalize

\def\data#1{\noindent\textcolor{blue}{\sffamily #1}}
\renewcommand{\descriptionlabel}[1]{\hspace{\labelsep}\textsf{\itshape #1}}
\newtheorem{teorema}{Teorema}[chapter]
\newtheorem{prop}[teorema]{Proposició}
\newtheorem{coro}[teorema]{Coro\l.lari}
\theoremstyle{definition}
\newtheorem{defi}[teorema]{Definició}
\newtheoremstyle{nota}{}{}{\sffamily}{}{\bfseries\sffamily\color{purple!85!black}}{.}{.5em}{}
\theoremstyle{nota}
\newtheorem{nota}[teorema]{Nota}
\newtheoremstyle{exemple}{}{}{\sffamily}{}{\bfseries\sffamily\color{green!40!black}}{.}{.5em}{}
\theoremstyle{exemple}
\newtheorem{exemple}[teorema]{Exemple}

\begin{document}

\makeatletter
\begin{titlepage}
  \noindent\Huge
  \begin{center}
    \textbf{\@title}
  \end{center}
  \vfill

  \noindent\large\sffamily
  Apunts de l'assignatura
  \hfill
  \@author
\end{titlepage}
\makeatother

% \chapter*{Nota preeliminar}
% \sffamily Aquests apunts segueixen el curs d'Anàlisi Matemàtica 3 del
% curs 2024 -- 2025 impartit per Javier Falcó Benavent.

\tableofcontents

\chapter{Preeliminars}

\section{Espais normats}

La mètrica usual en $\mathbb{R}$ és la distància eucídea entre dos
punts $x,\, y \in \mathbb{R}$, donada per $d(x,y) = |x-y|$. Aquest
concepte es generalitza a espais vectorials utilitzant la noció de
norma.

\begin{defi}[Espai normat]
  Un espai normat és un parell $(E, \|\cdot\|)$, on $E$ és un espai
  vectorial sobre el cos $\mathbb{K}$ ($\mathbb{R}$ o $\mathbb{C}$) i
  $\|\cdot\|$ és una funció anomenada norma, que satifà les següents
  propietats:
  \begin{description}[noitemsep]
  \item[Positivitat] $\|x\| \geq 0$, i $\|x\| = 0$ si, i només si,
    $x = 0$.
  \item[Homogeneïtat] $\|\lambda x\| = |\lambda| \|x\|$ per a tot
    $\lambda \in \mathbb{K}$ i $x \in E$.
  \item[Desigualtat triangular] $\|x + y\| \leq \|x\| + \|y\|$ per a
    tot $x,\, y \in E$.
  \end{description}
  Quan no hi haja ambigüitat, denotarem l'espai normat simplement com
  $E$.
\end{defi}

Un espai normat permet mesurar la longitud dels vectors i definir
conceptes com distància i convergència.

\begin{exemple}
  En l'espai $\mathbb{R}^n$, podem definir diferents normes. Les més
  comuns són
  \begin{align*}
    &\|x\|_1 = \sum_{i=1}^n|x_i|, &
    &\|x\|_2 = \Big( \sum_{i=1}^{n} |x_i|^2 \Big)^{\frac{1}{2}}, &
    &\|x\|_\infty = \max_{1 \leq 1 \leq n} |x_i|.
  \end{align*}
  No obstant això, també podem definir normes en espais més
  sofisticats, com és el cas de l'espai de les funcions contínues en
  un interval $[a,b]$ amb la norma
  \[
    \|f\| = \max_{x \in [a,b]} |f(x)|.
  \]
\end{exemple}

El concepte de norma ens recorda al concepte de mètrica. Recordem la
definició de mètrica sobre un conjunt $X$.

\begin{defi}[Mètrica]
  Una mètrica en un conjunt $X$ és una funció
  $d : X \times X \to \mathbb{R}$ que satisfà les següents propietats
  per a tot $x, \, y, \, z, \, \in X$:
  \begin{description}[noitemsep]
  \item[Positivitat] $d(x,y) \geq 0$, i $d(x,y) = 0$ si, i només si, $x = y$.
  \item[Simetria] $d(x,y) = d(y,x)$.
  \item[Desigualtat triangular] $d(x,y) \leq d(x,z) + d(z,y)$.
  \end{description}
\end{defi}

\begin{defi}[Mètrica induïda]
  La mètrica induïda per una norma $\|\cdot\|$ en un espai vectorial
  normat $(E, \|\cdot\|)$ es defineix com
  \[
    d(x,y) = \|x - y\|, \quad x,\, y \in E.
  \]
\end{defi}

\begin{nota}
  Encara que tota norma indueix una mètrica, no totes les mètriques
  provenen de normes. Un exemple clàssic de mètrica que no prové d'una
  norma és la mètrica discreta definida en qualsevol conjunt $X$,
  \[
    d(x,y) =
    \begin{cases}
      0 & \text{si}\; x = y, \\
      1 & \text{si}\; x \neq y.
    \end{cases}
  \]
  Una propietat destacada de la mètrica induïda per una norma és que
  aquesta és invariant per translacions. És a dir, per a tot
  $x,\, y,\, z \in E$, es compleix que
  \[
    d(x,y) = d(x+z, y+z).
  \]
\end{nota}

\begin{defi}[Bola]
  Siga $E$ un espai vectorial normat i $x_0 \in E$. Una bola oberta de
  centre $x_0$ i radi $r > 0$ és el conjunt
  \[
    B(x_0, r) = \{ x \in E : \|x - x_0\| < r \}.
  \]
\end{defi}

Notem que la bola oberta de radi $r$ centrada en $x_0$ coincideix amb
la bola oberta de radi $r$ centrada en l'origen traslladada per $x_0$,
\[
  B(x_0, r) = x_0 + B(0,r).
\]
Per tant, la bola oberta de radi $r$ centrada en l'origen és un
conjunt fonamental en la topologia d'un espai vectorial normat que
conté tota la informació topològica de l'espai.

\begin{figure}[htbp]
  \centering
  \begin{tikzpicture}[scale=0.8]
    \draw[thick,->] (-1.5,0) -- (1.5,0);
    \draw[thick,->] (0,-1.5) -- (0,1.5);
    \fill[orange!30, fill opacity=0.5] (1,0) -- (0,1) -- (-1,0) -- (0,-1) -- cycle;

    \draw[orange, thick] (1,0) -- (0,1) -- (-1,0) -- (0,-1) -- cycle;
    \node at (-1.2,1.4) {$\|\cdot\|_1$};
    \draw[thick,->] (3,0) -- (6,0);
    \draw[thick,->] (4.5,-1.5) -- (4.5,1.5);
    \fill[orange!30, fill opacity=0.5] (4.5,0) circle(1);
    \draw[orange, thick] (4.5,0) circle(1);
    \node at (3.3,1.4) {$\|\cdot\|_2$};

    \draw[thick,->] (7.5,0) -- (10.5,0);
    \draw[thick,->] (9,-1.5) -- (9,1.5);
    \fill[orange!30, fill opacity=0.5] (8,1) rectangle (10,-1);
    \draw[orange, thick] (8,1) rectangle (10,-1);
    \node at (7.8,1.4) {$\|\cdot\|_\infty$};
  \end{tikzpicture}
  \caption{$B(0,1)$ en diferents normes.}
  \label{fig:bola-normes}
\end{figure}

\begin{prop}[Propietats addicionals de la norma]
  Siga $E$ un espai vectorial normat i siguen $x, \, y \in E$,
  aleshores tenim les següents propietats.
  \begin{description}[noitemsep]
  \item[Continuïtat] $\|\cdot\|$ és contínua respecte a la topologia
    induïda per la mètrica $d(x,y) = \|x - y\|$.
  \item[Convexitat de boles] Tota bola oberta
    $B(x_0, r) = \{y \in E : \|y - x_0\| < r\}$ és un conjunt convex.
  \item[Escalabilitat] Si $\|x\| = 1$, aleshores
    $\|\lambda x\| = |\lambda|$ per a tot $\lambda \in \mathbb{K}$.
  \end{description}
\end{prop}

Com hem mencionat, la bola oberta de radi $r$ centrada en l'origne
conté tota la informació topològica de l'espai. Per tant, la primera
propietat de la proposició anterior pot millorar-se obtenint que la
norma és uniformement contínua en l'espai.

Recordem que una funció $f : X \to Y$ és uniformement contínua si la
variació de la funció no depén de l'elecció dels punts en el
domini. És a dir, és uniformementcontínua si donat un $\epsilon > 0$
existeix un $\delta > 0$ tal que si $d(x,y) < \delta$, aleshores
$d(f(x), f(y)) < \epsilon$.

\begin{prop}[Continuïtat uniforme de la norma]
  La norma $\|\cdot\|$ és uniformement contínua en $E$.
\end{prop}

\begin{proof}
  Donats $x, \, y \in E$, per la desigualtat triangular tenim
  \begin{equation*}
    \left.
      \begin{aligned}
        \|x\| &\leq \|x-y\| - \|y\| & \implies \|x\|-\|y\| &\leq \|x-y\| \\
        \|y\| &\leq \|x-y\| - \|x\| & \implies \|y\|-\|x\| &\leq \|x-y\|
      \end{aligned} \;\;
    \right\} \;\;
    \left| \|x\| - \|y\| \right| \leq \|x-y\|.
  \end{equation*}
  Recordem que $\|\cdot\| : E \to \mathbb{R}^+ \cup \{0\}$. Donat
  $\epsilon > 0$, considerem $\delta = \epsilon$. Aleshores,
  $\left| \|x\| - \|y\| \right| \leq \|x - y\| \leq \delta = \epsilon$
  per a tot $\|x-y\| \leq \delta$. És a dir, la norma és contínuament
  uniforme.
\end{proof}

Recordem que una aplicació lineal entre espais vectorials és una
aplicació que respecta l'estructura d'espai vectorial, és a dir, les
operacions de suma de vectors i producte per escalars.

\begin{defi}[Aplicació lineal]
  Siguen $E$ i $F$ dos espais vectorials sobre un camp
  $\mathbb{K}$. Una aplicació $T : E \to F$ es diu que és lineal si
  per a tot $x, \, y \in E$ i tot $\lambda \in \mathbb{K}$ es compleix
  que
  \begin{enumerate}[nosep]
  \item $T(x + y) = T(x) + T(y)$,
  \item $T(\lambda x) = \lambda T(x)$.
  \end{enumerate}
\end{defi}

\begin{teorema}[Caracterització de la continuïtat d'aplicacions lineals]
  \label{teo:caracterització_continuïtat}
  Siguen $(E, \|\cdot\|_E)$ i $(F, \|\cdot\|_F)$ dos espais normats i
  $T : E \to F$ una aplicació lineal. Aleshores, les següents
  afirmacions són equivalents.
  \begin{enumerate}
  \item\label{c_cont:1} $T$ és contínua en $E$.
  \item\label{c_cont:2} $T$ és contínua en l'origen de $E$.
  \item\label{c_cont:3} Existeix una constant $C > 0$ tal que per a
    tot $x \in E$ es compleix que $\|T(x)\|_F \leq C \|x\|_E$.
  \item\label{c_cont:4} Existeix una constant $C > 0$ tal que per a
    tot $x \in B_E$ es compleix que $\|T(x)\|_F \leq C$.
  \item\label{c_cont:5} El conjunt $T(B_E)$ és fitat en $F$.
  \end{enumerate}
\end{teorema}

\begin{proof}
  La implicació $(\ref{c_cont:1} \Rightarrow \ref{c_cont:2})$ és
  directa. Per a $(\ref{c_cont:2} \Rightarrow \ref{c_cont:3})$
  observem que $T$ és contínua en $0$ i és lineal. Per tant,
  $T(0) = 0$ i existeix $r > 0$ tal que
  $B_E(0,r) \subset T^{-1}(B_F(0,1))$. Per tant, per a tot
  $x \in B_E(0,r)$ tenim que $T(x) \in B_F(0,1)$. Així,
  $\|T(x)\|_F \leq 1$, però
  \[
    \|T(x)\|_F = \frac{\|x\|_E}{r} \underbrace{\left\| \,
        T\left(\frac{rx}{\|x\|_E}\right) \, \right\|_F}_{\leq 1} \leq
    \frac{\|x\|_E}{r}.
  \]
  Així, considerant $C = \frac{1}{r}$ tenim el que volíem demostrar.
  $(\ref{c_cont:3} \Rightarrow \ref{c_cont:4} \Rightarrow
  \ref{c_cont:5})$ és directe i sols queda demostrar
  $(\ref{c_cont:5} \Rightarrow \ref{c_cont:1})$. Que $T(B_E)$ siga
  fitat implica que existeix $C > 0$ amb $\|T(x)\| \leq C$ per a tot
  $x \in B_E$. Fixat $x_0 \in E$ i donat $\epsilon > 0$, considerem
  $\delta = \frac{\epsilon}{C}$. Aleshores si $y \in E$ amb
  $\|x_0 - y\| < \delta$,
  \begin{align*}
    \|T(x_0) - T(y)\|_F
    &= \|T(x_0 - y)\| =
      \left\| \|x_0-y\| \, T \left( \frac{x_0 - y}{\|x_0 - y\|}\right) \right\| \\
    &\leq \|x_0 - y\| \cdot C \leq \delta \cdot C = \frac{\epsilon}{C} \cdot C = \epsilon.\qedhere
  \end{align*}
\end{proof}

\begin{nota}
  La condició 3 es pot reescriure com que $T$ és Lipschitz contínua.
\end{nota}

\begin{exemple}
  Per a cada funció $\varphi : [a,b] \to [a,b]$ contínua, l'aplicació
  \begin{align*}
    T : \mathcal C([a,b]) &\longrightarrow \mathcal C([a,b]), \\
    T (f) &\longmapsto f \circ \varphi, 
  \end{align*}
  és contínua.
\end{exemple}

\begin{proof}
  Per a vore que $T$ és contínua, cal provar que és lineal. Observem
  que
  \[
    \|T(f)\|_\infty = \sup_{x\in[a,b]} \left|(f\circ\varphi)(x)\right|
    \leq \sup_{x\in[a,b]}|f(x)| = \|f\|_\infty.
  \]
  Pel cas \ref{c_cont:3} del Teorema
  \ref{teo:caracterització_continuïtat} amb $C=1$, tenim que $T$ és
  contínua.
\end{proof}

\begin{coro}
  En tot espai normat $E$ de dimensió infinita existeixen formes
  lineals que no són contínues.
\end{coro}

\begin{proof}
  Com $E$ té dimensió infinita, existeix
  $(l_n)_{n=1}^\infty \subset E$, vectors linealment independents amb
  $\|l_n\| = 1$. Considerem $\mathcal{B}$ una base de $E$ amb
  $(l_n)_{n=1}^\infty \subset \mathcal{B}$ (assumint l'axioma
  d'elecció). Reescrivim $\mathcal{B} = \{e_i\}_{i \in I}$,
  $\mathbb{N} \subset I$ i definim
  \begin{equation*}
    \begin{array}{lrcl}
      f : & E &\longrightarrow& E, \\
          & x = \sum\limits_{i \in I}\alpha_ie_i
              &\longmapsto& f(x) = \sum\limits_{n=1}^{\infty} ne_n.
    \end{array}
  \end{equation*}
  Així, $f(e_n) = ne_n$ i $\|f(e_n)\| = \|ne_n\| = n$. Per tant, $f$
  no està fitada i no és contínua.
\end{proof}

\section{Espais de Banach}

Recordem algunes definicions bàsiques de l'anàlisi matemàtica en el
context d'espais normats.

\begin{defi}[Successió de Cauchy]
  Una successió $\{x_n\}_{n \in \mathbb{N}}$ en un espai vectorial
  normat $E$ es diu que és de Cauchy si per a tot $\epsilon > 0$
  existeix $N \in \mathbb{N}$ tal que per a tot $n, \, m \geq N$ es
  compleix que $\|x_n - x_m\| < \epsilon$.
\end{defi}

\begin{exemple}
  En l'espai $E = (\mathcal{C}[0,1], \|\cdot\|_\infty)$ de les
  funcions contínues en l'interval $[0,1]$ amb la norma del suprem, la
  successió de funcions $\{f_n\}_{n \in \mathbb{N}}$ definida per
  $f_n(x) = \left( \frac{x}{2} \right)^n$ és de Cauchy.
\end{exemple}

\begin{defi}[Successió convergent]
  Una successió $\{x_n\}_{n \in \mathbb{N}}$ en un espai vectorial
  normat $E$ es diu que és convergent a un punt $x \in E$ si per a tot
  $\epsilon > 0$ existeix un $N \in \mathbb{N}$ tal que per a tot
  $n \geq N$ es compleix que $\|x_n - x\| < \epsilon$.
\end{defi}

Aquesta definició es pot expressar en termes de límits com
\[
  \lim_{n \to \infty} x_n = x \quad \iff \quad
  \lim_{n \to \infty} \|x_n - x\| = 0 \quad \iff \quad
  \lim_{n \to \infty} d(x_n, x) = 0.
\]

\begin{defi}[Convergència de sèries]
  Una sèrie $\sum_{n=1}^{\infty} x_n$ en un espai vectorial normat $E$
  es diu que és convergent a un punt $x \in E$ si la successió de
  sumes parcials $\{s_n\}_{n \in \mathbb{N}}$ definida per
  $s_n = \sum_{k=1}^{n} x_k$ és convergent a $x$, és a dir, si
  \[
    \lim_{n \to \infty} \Big\| \, x - \sum_{k=1}^{n} x_k \, \Big\| = 0.
  \]
\end{defi}

\begin{defi}[Espai de Banach]
  Un espai normat $E$ es diu que és un espai de Banach si és complet
  respecte a la norma $\|\cdot\|$, és a dir, si tota successió de
  Cauchy en $E$ convergeix a un punt en $E$.
\end{defi}

\begin{exemple}
  \begin{itemize}
  \item Un exemple d'espai normat que és un espai de Banach són els
    nombres reals $\mathbb{R}$ amb la norma definida pel valor absolut.
  \item Un exemple d'espai normat que no és un espai de Banach és
    l'espai de les successions de suport finit amb la norma suprem. És
    a dir, el conjunt
    \[
      c_{00} = \left\{ x = (x_1, x_2, \dotsc) : \exists\, N \in \mathbb{N}
        \;\; \text{tal que} \;\; x_n = 0, \;\; \forall\, n > N \right\}
    \]
    amb la norma
    \[
      \|x\|_\infty = \sup_{i \in \mathbb{N}} |x_i|
    \]
    no és un espai de Banach.
  \end{itemize}
\end{exemple}

Altres exemples d'espais de Banach són els espais de funcions
contínues i els espais de funcions integrables.

\begin{exemple}
  Siga $K$ un espai topològic compacte, denotem per
  $(\mathcal{C}(K), \|\cdot\|_\infty)$ a l'espai de les funcions
  contínues en $K$ amb la norma $\|f\| = \sup_{x \in K}
  |f(x)|$. Aquest espai és un espai de Banach.
\end{exemple}

Sabem que tota sèrie de nombres reals o complexos que és absolutament
convergent és convergent. Vegem que aquest resultat també es vàlid per
a espais de Banach.

\begin{teorema}
  Siga $(E, \|\cdot\|)$ un espai normat. Aleshores, les següents
  afirmacions són equivalents.
  \begin{enumerate}[(i), nosep]
  \item $(E, \|\cdot\|)$ és un espai de Banach.
  \item Si $(x_n)_n \subset E$ amb
    $\sum_{n=1}^{\infty} \|x_n\| < \infty$, aleshores la sèrie
    $\sum_{n=1}^{\infty} x_n$ és convergent en $E$.
  \end{enumerate}
\end{teorema}

\section{Espais normats de dimensió finita}

El nostre següent objectiu és definir una família de normes en
$\mathbb{R}^n$ (o $\mathbb{C}^n$) que generalitzen la norma
euclidiana. Per a això, considerem la norma $\|x\|_p$ definida per
\[
  \|x\|_p = \Big( \sum_{i=1}^{n} |x_i|^p \Big)^{\frac{1}{p}},
  \quad x = (x_1, x_2, \dotsc, x_n) \in \mathbb{R}^n.
\]
Per a poder demostrar que $\|\cdot\|_p$ és una norma, necessitem
demostrar que satisfà les propietats de positivitat, homogeneïtat i
desigualtat triangular. La positivitat i l'homogeneïtat són propietats
immediates de la definició de $\|\cdot\|_p$, per la qual cosa ens
enfocarem a demostrar la desigualtat triangular. Per poder demostrar
la desigualtat triangular, necessitem fer ús de les següents tres
desigualtats fonamentals: la desigualtat de Young, la desigualtat de
Hölder i la desigualtat de Minkowski.

Recordem que una funció $f : \mathbb{R} \to \mathbb{R}$ és convexa si
per a tot $x, \, y \in \mathbb{R}$ i per a tot $\lambda \in [0,1]$ es
compleix que
\[
  f(\lambda x + (1-\lambda) y) \leq \lambda f(x) + (1-\lambda) f(y).
\]
Gràficament, una funció és convexa si la seua gràfica es troba per
davall de la recta que uneix dos punts qualssevol de la gràfica.

\begin{defi}[Exponent conjugat]
  L'exponent conjugat d'un nombre $p \in (1, +\infty)$ es defineix com
  el nombre $q \in (1, +\infty)$ tal que
  \[
    \frac{1}{p} + \frac{1}{q} = 1.
  \]
  En el cas que $p = 1$ o $p = +\infty$, es defineix $q = +\infty$ o
  $q = 1$, respectivament.
\end{defi}

\begin{exemple}
  \begin{itemize}[nosep]
  \item Si $p=3$, aleshores $q = \frac{3}{2}$.
  \item Si $p = \sqrt{2}$, aleshores $q = 2 + \sqrt{2}$.
  \item Si $p = 2$, aleshores $q = 2$.
  \end{itemize}
\end{exemple}

Entre les propietats que es poden deduir de l'exponent conjugat, tenim
la següent.

\begin{prop}
  Si $p,\, q > 1$ són tals que $q$ és l'exponent conjugat de $p$,
  aleshores $pq = p + q$.
\end{prop}

\begin{teorema}[Desigualtat de Young]
  Siguen $a,\, b \in \mathbb{R}$ i $p,\, q > 1$ tals que $q$ és
  l'exponent conjugat de $p$, és a dir
  $\frac{1}{p} + \frac{1}{q} = 1$, aleshores
  \[
    ab \leq \frac{a^p}{p} + \frac{b^q}{q}.
  \]
\end{teorema}

\begin{proof}
  Per a provar aquest resultat gastarem la convexitat de la funció
  exponencial. Així,
  \begin{align*}
    ab &= e^{\ln(ab)} = e^{\ln(a) + \ln(b)}
         = e^{p \frac{\ln(a)}{p} + q \frac{\ln(b)}{q}}
         = e^{\frac{\ln(a^p)}{p}+\frac{\ln(b^q)}{q}} \\
       &\leq \frac{1}{p} e^{\ln(a^p)} + \frac{1}{q} e^{\ln(b^q)}
         = \frac{a^p}{p} + \frac{b^q}{q}.\qedhere
  \end{align*}
\end{proof}

La desigualtat té la següent interpretació geomètrica. Si considerem
les funcions
\[
  \phi(t) = t^{p-1}, \quad \psi(t) = t^{q-1},
\]
aleshores $\phi(t)$ i $\psi(t)$ són funcions tals que una és la
inversa de l'altra. És a dir, si $\phi(t) = a$, aleshores
$\psi(a) = t$ i viceversa. La desigualtat de Young ens diu que l'àrea
del rectangle de costats $a$ i $b$ és menor o igual a la suma de les
àrees sota les corbes $\phi(t)$ i $\psi(t)$. Veure la Figura
\ref{fig:young}.

\begin{figure}[htbp]
  \centering
  \begin{tikzpicture}
    \def\a{2.25}
    \def\b{2}
    \def\c{4}
    \def\p{1.5}
    
    \draw[->] (-0.5,0) -- (5,0) node[right] {\footnotesize $x$};
    \draw[->] (0,-0.5) -- (0,2.75) node[above] {\footnotesize $y$};
    
    \fill[Violet!20] plot[domain=0:\c, samples=100] (\x,{\x^(\p-1)}) -- (0,\b) -- cycle;
    \fill[MidnightBlue!20] plot[domain=0:\a, samples=100] (\x,{\x^(\p-1)}) -- (\a,0) -- cycle;
    
    \draw[red,dashed,thick] (0,\b) rectangle (\a,0);
    \draw[blue,thick,domain=0:4.5,samples=100] plot(\x,{\x^(\p-1)}) node[right,blue] {\footnotesize $y=\phi(x)$};
    \node[left] at (0,\b) {\textit{b}};
    \node[below] at (\a,0) {\textit{a}};
  \end{tikzpicture}
  \caption{Interpretació geomètrica de la desigualtat de Young.}
  \label{fig:young}
\end{figure}

La desigualtat de Hölder és una conseqüència de la desigualtat de
Young que ens permet fitar el producte de dues funcions en termes de
les seues normes.

\begin{teorema}[Desigualtat de Hölder]
  Siguen $p,\, q > 1$ tals que $\frac{1}{p} + \frac{1}{q} = 1$, i
  siguen $x_i,\, y_i \in \mathbb{R}$ per a $i = 1, 2, \dotsc, n$
  nombres reals no negatius, aleshores
  \[
    \sum_{i=1}^{n} x_i y_i \leq
    \Big( \sum_{i=1}^{n} x_i^p \Big)^{\frac{1}{p}}
    \Big( \sum_{i=1}^{n} y_i^q \Big)^{\frac{1}{q}}.
  \]
\end{teorema}

Notem que en la desigualtat anterior podem considerar
$x_i, \, y_i \in \mathbb{C}$ per a $i = 1, 2, \dotsc, n$, i la
desigualtat continua sent vàlida.

La desigualtat de Minkowski ens permet fitar la norma de la suma de
dos vectors en $\mathbb{R}^n$.

\begin{teorema}[Desigualtat de Minkowski]
  Siguen $p > 1$ i $x,\, y \in \mathbb{R}^n$, aleshores
  \[
    \Big( \sum_{i=1}^{n} |x_i + y_i|^p \Big)^{\frac{1}{p}} \leq
    \Big( \sum_{i=1}^{n} |x_i|^p \Big)^{\frac{1}{p}} +
    \Big( \sum_{i=1}^{n} |y_i|^p \Big)^{\frac{1}{p}}.
  \]
\end{teorema}

Ara que hem demostrat les desigualtats de Hölder i Minkowski, podem
demostrar que l'espai $\mathbb{K}^n$ és un espai normat amb
$\|\cdot\|_p$.

\begin{teorema}
  Per a tot nombre $1 \leq p < + \infty$, l'espai $\mathbb{K}^n$ és un
  espai normat amb la norma
  \[
    \|x\|_p = \Big( \sum_{i=1}^{n} |x_i|^p \Big)^{\frac{1}{p}}.
  \]
\end{teorema}

\begin{itemize}
\item Per a $p = 2$ obtenim l'espai euclidià $\mathbb{R}^n$ amb la
  norma euclidiana $\|x\|_2 = \sqrt{x_1^2 + x_2^2 + \dotsb + x_n^2}$.
\item Per a $p = 1$ obtenim l'espai de les sumes absolutes
  $\|x\|_1 = |x_1| + |x_2| + \dotsb + |x_n|$.
\item Per a $p = \infty$, la norma
  $\|x\|_\infty = \max_{i=1,\dotsc,n}|x_i|$ és la norma del màxim i
  l'espai $\ell^\infty(\mathbb{R}^n)$ és també un espai normat.
\end{itemize}

\begin{defi}[Normes equivalents]
  Dues normes $\|\cdot\|_1$ i $\|\cdot\|_2$ en un espai vectorial $E$
  es diuen equivalents si exisiteixen constants $C_1,\, C_2 > 0$ tals
  que
  \[
    C_1 \|x\|_1 \leq \|x\|_2 \leq C_2 \|x\|_1, \quad \forall\, x \in E.
  \]
\end{defi}

\begin{teorema}[Equivalència de normes en $\mathbb{K}^n$]
  Siga $n \in \mathbb{N}$, aleshores totes les normes en
  $\mathbb{K}^n$ són equivalents.
\end{teorema}

És important remarcar el següent resultat que no demostrarem en aquest
curs. La demostració d'aquest resultat s'estudia al curs d'Anàlisi
Funcional.

\begin{teorema}[Compacitat de la bola unitat]
  Siga $E$ un espai normat. Aleshores, les següents afirmacions són
  equivalents.
  \begin{enumerate}[nosep]
  \item $E$ té dimensió finita.
  \item La bola unitat tancada d'$E$ és compacta.
  \item Tot subconjunt d'$E$ tancat i fitat és compacte.
  \end{enumerate}
\end{teorema}

\section{Els espais \texorpdfstring{$\ell^p$}{lp}}

\begin{defi}[Espais $\ell^p$]
  Per a $1 \leq p < \infty$, l'espai $\ell^p$ es defineix com l'espai
  de les successions $x = (x_1, x_2, \dotsc)$ amb la $p$-norma fitada,
  \[
    \ell^p = \left\{ x = (x_1, x_2, \dotsc) : \Big(
      \sum_{i=1}^{\infty} |x_i|^p \Big)^{\frac{1}{p}} < \infty
    \right\}.
  \]
  Per a $p = \infty$, l'espai $\ell^\infty$ es defineix com l'espai de
  les successions $x = (x_1, x_2, \dotsc)$ fitades amb la norma del
  suprem,
  \[
    \ell^\infty = \left\{ x = (x_1, x_2, \dotsc) : \sup_{i \in
        \mathbb{N}} |x_i| < \infty \right\}.
  \]
  En tots els casos, l'espai $(\ell^p, \|\cdot\|_p)$ és un espai
  normat.
\end{defi}

\begin{teorema}
  Per a $1 \leq p \leq \infty$, l'espai $\ell^p$ és un espai de
  Banach.
\end{teorema}

\begin{proof}
  Siga $1 \leq p \leq \infty$. Considerem $(X^{(n)})_{n=1}^\infty$ de
  Cauchy. Aleshores, fixat $\epsilon > 0$, existeix $N \in \mathbb{N}$
  tal que
  \[
    \|X^{(n)}-X^{(m)}\|_p < \epsilon, \quad \forall\,n,\,m > N.
  \]
  Per a cada $k \in \mathbb{N}$ tenim que
  \[
    |X_k^{(n)}-X_k^{(m)}| \leq \|X_k^{(n)}-X_k^{(m)}\|_p < \epsilon,
    \quad \forall\,n,\ m > N.
  \]
  Per tant, $(X_k^{(n)})_{k=1}^\infty$ és una successió de Cauchy en
  $\mathbb{K}$. Per tant, existeix el límit
  \[
    \lim_{n\to\infty} X_k^{(n)} = X_k.
  \]
  Definim $X = (X_1, X_2, \dotsc)$. Notem que per a tot
  $m \in \mathbb{N}$ tenim que
  \[
    \sum_{k=1}^m |X_k^{(n)}-X_k^{(j)}|^p \leq
    \sum_{k=1}^{\infty}|X_k^{(n)} - X_k^{(j)}|^p = \|X^{(n)} -
    X^{(j)}\|_p^p < \epsilon^p \quad \text{si} \;\; n,\,j > N.
  \]
  Prenent límits quan $j \to \infty$,
  \[
    \sum_{k=1}^{m}|X_k^{(n)} - X_k^{(j)}|^p < \epsilon^p, \quad
    \forall\,m \in \mathbb{N},\;\; \forall\,n > N.
  \]
  Per tant, apleguem a
  \[
    \|X^{(n)}-X\|_p^p = \sum_{k=1}^{\infty} |X_k^{(n)}-X_k|^p <
    \epsilon^p, \quad \forall \, n > N.
  \]
  Per definició $X^{(n)} \to X$ en $\|\cdot\|_p$. Per últim, aplicant
  la desigualtat triangular,
  \[
    \Big( \sum_{i=1}^\infty |X_i|^p \Big)^{\frac{1}{p}} \leq
    \Big( \sum_{i=1}^\infty |X_i - X_i^{(n)}|^p \Big)^{\frac{1}{p}} +
    \Big( \sum_{i=1}^\infty |X_i^{(n)}|^p \Big)^{\frac{1}{p}}
    \quad \forall \, n \in \mathbb{N}.
  \]
  Considerant $n$ tal que $\|X-X^{(n)}\|_p < 1$, tenim que
  $\|X\|_p \leq 1 + \|X^{(n)}\| < +\infty$. Per tant, $X \in \ell^p$.
\end{proof}

Els espais $\ell^p$ tenen una relació d'inclusió entre ells com
segueix.

\begin{teorema}
  Per a $1 \leq p < q \leq \infty$, es compleix que
  $\ell^p \subset l^q$.
\end{teorema}

\section{Espais de Lebesgue \texorpdfstring{$L^p(\Omega)$}{Lp}}

Siga $\Omega$ un conjunt mesurable i $\mu$ una mesura en $\Omega$. Per
a $1 \leq p < \infty$, l'espai $\mathcal{L}^p(\Omega)$ es defineix com
l'espai de les funcions mesurables $f : \Omega \to \mathbb{R}$ tals
que
\[
  \|f\|_p = \left( \int_\Omega |f(x)|^p \, \mathrm{d}\mu(x)
  \right)^{\frac{1}{p}} < \infty,
\]
és a dir,
\[
  \mathcal{L}^p(\Omega) = \left\{ f : \Omega \to \mathbb{R}\ :\ \left(
      \int_\Omega |f(x)|^p \, \mathrm{d}\mu(x) \right)^{\frac{1}{p}} <
    \infty \right\}.
\]
És evident que $\mathcal{L}^p(\Omega)$ amb la funció $\|\cdot\|_p$
verifica la propietat d'homogeneïtat de la definició de norma per la
linealitat de la integral. Per a veure que $\|\cdot\|_p$ verifica la
desigualtat triangular, necessitem la versió de les desigualtats de
Hölder i Minkowski per a integrals.

\begin{teorema}[Desigualtat de Hölder per a integrals]
  Siguen $p,\,q>1$ tals que $\frac{1}{p} + \frac{1}{q} = 1$, i siguen
  $f \in \mathcal{L}^p(\Omega)$ i $g \in \mathcal{L}^q(\Omega)$,
  aleshores $fg \in \mathcal{L}^1(\Omega)$ i
  \[
    \left| \int_\Omega f(x)g(x)\, \mathrm{d}x \right| \leq
    \int_\Omega |f(x)g(x)|\,\mathrm{d}x \leq
    \left(\int_\Omega |f(x)|^p\,\mathrm{d}x\right)^{\frac{1}{p}}
    \left(\int_\Omega |g(x)|^q\,\mathrm{d}x\right)^{\frac{1}{q}}.
  \]
\end{teorema}

\begin{proof}
  Suposem $\|f\|_p, \, \|g\|_q \neq 0$. Que el mòdul de la integral és
  major o igual que la integral del mòdul es té per les propietats de
  la integral de Lebesgue. Per tant, sols hem de demostrar la segona
  desigualtat. Per a cada $X \in \Omega$, per la desigualtat de Young,
  \[
    \frac{|f(x)|}{\|f\|_p} \cdot \frac{|g(x)|}{\|g\|_q} \leq
    \frac{\left( \frac{|f(x)|}{\|f\|_p} \right)^p}{p} +
    \frac{\left( \frac{|g(x)|}{\|g\|_q} \right)^q}{q} =
    \frac{|f(x)|^p}{p\|f\|_p^p} + \frac{|g(x)|^q}{q\|g\|_q^q}.
  \]
  Per tant,
  \begin{align*}
    \int_\Omega \frac{|f(x)|}{\|f\|_p} \cdot \frac{|g(x)|}{\|g\|_q}
    \,\mathrm{d}x
    &\leq \int_\Omega \frac{|f(x)|^p}{p\|f\|_p^p} +
      \frac{|g(x)|^q}{q\|g\|_q^q} \,\mathrm{d}x \\
    &=\frac{1}{p} \cdot
      \underbrace{\frac{\int_\Omega|f(x)|^p\,\mathrm{d}x}{\|f\|_p^p}}_1 +
      \frac{1}{q} \cdot
      \underbrace{\frac{\int_\Omega|g(x)|^q\,\mathrm{d}x}{\|g\|_q^q}}_1
      = \frac{1}{p} + \frac{1}{q} = 1.
  \end{align*}
  Així, tenim que $fg \in \mathcal{L}^1(\Omega)$ i 
  \(
    \int_\Omega |f(x)g(x)| \, \mathrm{d}x \leq \|f\|_p \cdot \|g\|_q.
  \)
\end{proof}

\begin{teorema}[Desigualtat de Minkowski per a integrals]
  Siguen $p \geq 1$ i $f,\,g : \Omega \to \mathbb{R}$ són funcions en
  $\mathcal{L}^p(\Omega)$, aleshores $f+g \in \mathcal{L}^p(\Omega)$ i
  \[
    \|f + g\|_p \leq \|f\|_p + \|g\|_p.
  \]
\end{teorema}

\begin{proof}
  Per a $p=1$ el resultat és directe. Suposem que $p > 1$ i vegem que
  $f + g \in L^p(\Omega)$. Per a vore aquesta part, notem que per a
  tot $x \in \Omega$,
  \[
    |f(x) + g(x)|^p \leq 2^p \max\{|f(x)|^p,\,|g(x)|^p\}
    \leq 2^p (|f(x)|^p + |g(x)|^p).
  \]
  Per tant,
  \begin{align*}
    \int_\Omega |f(x) + g(x)|^p\,\mathrm{d}x
    &\leq 2^p \left( \int_\Omega |f(x)|^p \, \mathrm{d}x +
      \int_\Omega |g(x)|^p \, \mathrm{d}x \right) \\
    &= 2^p \big( \|f\|_p^p + \|g\|_p^p \big) < + \infty.
  \end{align*}
  Aleshores, $f + g \in L^p(\Omega)$. Queda comprovar que
  $\|f + g\|_p \leq \|f\|_p + \|g\|_p$. Observem que
  \begin{align*}
    \|f+g\|_p^p
    &= \int_\Omega |f(x) + g(x)|^p \, \mathrm{d}x \\
    &= \int_\Omega \textcolor{teal}{|f(x) + g(x)|^{p-1}}
      \textcolor{violet}{f(x)} \, \mathrm{d}x +
      \int_\Omega \textcolor{teal}{|f(x) + g(x)|^{p-1}}
      \textcolor{violet}{g(x)} \, \mathrm{d}x \\
    &\overset{\mathclap{\text{Hölder}}}{\leq}\;\;\,
      \textcolor{violet}{\|f\|_p}
      \textcolor{teal}{\|(f+g)^{p-1}\|_q} +
      \textcolor{violet}{\|g\|_p}
      \textcolor{teal}{\|(f+g)^{p-1}\|_q} \\
    &= \textcolor{violet}{\|f\|_p}
      \textcolor{teal}{\|f+g\|_p^{p-1}} +
      \textcolor{violet}{\|g\|_p}
      \textcolor{teal}{\|f+g\|_p^{p-1}}
  \end{align*}
  i concloem que $\|f+g\|_p \leq \|f\|_p + \|g\|_p$.
\end{proof}

Per tant, el conjunt $\mathcal{L}^p(\Omega)$ amb la funció
$\|\cdot\|_p$ verifica les propietats d'homogeneïtat i desigualtat
triangular, de la definició d'espai normat. No obstant això, encara
que $\|f\| \geq 0$ per a tota funció $f \in \mathcal{L}^p(\Omega)$, no
es verifica la propietat de positivitat, ja que la norma $\|f\|_p$ pot
ser zero encara que la funció $f$ no siga la funció nu\l.la.

Per a aconseguir que es verifique aquesta propietat, podem realitzar
una relació d'equivalència en $\mathcal{L}^p(\Omega)$ on dues funcions
$f, \, g \in \mathcal{L}^p(\Omega)$ són equivalents si $f = g$ quasi
per totes parts, és a dir, si $f(x) = g(x)$ per a tot $x \in \Omega$
excepte en un conjunt de mesura zero.

L'espai quocient $\mathcal{L}^p(\Omega)/\sim$ amb la norma
$\|\cdot\|_p$ és l'espai de Lebesgue $L^p(\Omega)$.

\begin{defi}[Espais de Lebesgue $L^p(\Omega)$]
  Per a $1 \leq p < \infty$, l'espai de Lebesgue $L^p(\Omega)$ es
  defineix com l'espai quocient $\mathcal{L}^p(\Omega) / \sim$, amb la
  norma
  \[
    \|f\|_p = \left( \int_\Omega |f(x)|^p \, \mathrm{d}\mu(x)
    \right)^{\frac{1}{p}}.
  \]
  En tots els casos, l'espai $(L^p(\Omega), \|\cdot\|_p)$ és un espai
  de Banach.
\end{defi}

\begin{exemple}
  Un exemple d'espai de Lebesgue és l'espai $L^2([a,b])$ de les
  funcions quadrat integrables en l'interval $[a,b]$ amb la norma
  \[
    \|f\|_2 = \left( \int_a^b |f(x)|^2 \, \mathrm{d}x \right)^{\frac{1}{2}}.
  \]
  Notem que en els espais $L^p(\Omega)$ la convergència en la norma
  $\|\cdot\|_p$ no implica la convergència puntual de la successió de
  funcions.
\end{exemple}

\begin{exemple}
  Considerem la successió de funcions $(f_n)_n \subset L^p([0,1])$
  definida com
  \begin{align*}
    &f_1 = \chi_{\left[0,\frac{1}{2}\right]}, &
    &f_2 = \chi_{\left[\frac{1}{2},1\right]}, &
    &f_3 = \chi_{\left[0,\frac{1}{4}\right]}, &
    &f_4 = \chi_{\left[\frac{1}{4},\frac{1}{2}\right]}, &
    &\dotsc
  \end{align*}
  Mostrem la seua representació gràfica a la Figura
  \ref{fig:exemple-no-conver}. Aleshores, tenim que
  \[
    \|f_n - 0\|_p = \int_0^1 \chi_{[a,b]}(x) \, \mathrm{d}x
    \;\; \xrightarrow{n \to \infty} \;\; 0.
  \]

  No obstant això, per a tot punt $x \in [0,1]$, la successió
  $(f_n(x))_n$ pren els valors $0$ i $1$ infinites vegades, per la
  qual cosa aquesta successió $(f_n(x))_n$ no convergeix a cap punt
  $x \in [0,1]$.
\end{exemple}

\begin{figure}[htbp]
  \centering
  \begin{tikzpicture}[scale=1.7]
    \def\colfun{NavyBlue}
    \def\relleno{NavyBlue!30}
    
    % Fila 1
    \draw[->] (-0.2,2) -- (1.2,2);
    \draw[->] (0,1.8) -- (0,3.2);
    \fill[fill opacity=.5, \relleno] (0,3) rectangle (0.5,2);
    \draw[thin, dashed] (0.5,2) -- (0.5,3);
    \draw[very thick, \colfun] (0,3) -- (0.5,3);
    \draw[very thick, \colfun] (0.5,2) -- (1,2);

    \draw[->] (1.8,2) -- (3.2,2);
    \draw[->] (2,1.8) -- (2,3.2);
    \fill[fill opacity=.5, \relleno] (2.5,3) rectangle (3,2);
    \draw[thin, dashed] (2.5,2) -- (2.5,3);
    \draw[thin, dashed] (3,2) -- (3,3);
    \draw[very thick, \colfun] (2,2) -- (2.5,2);
    \draw[very thick, \colfun] (2.5,3) -- (3,3);

    \draw[->] (3.8,2) -- (5.2,2);
    \draw[->] (4,1.8) -- (4,3.2);
    \fill[fill opacity=.5, \relleno] (4,3) rectangle (4.25,2);
    \draw[thin, dashed] (4.25,2) -- (4.25,3);
    \draw[very thick, \colfun] (4,3) -- (4.25,3);
    \draw[very thick, \colfun] (4.25,2) -- (5,2);

    \draw[->] (5.8,2) -- (7.2,2);
    \draw[->] (6,1.8) -- (6,3.2);
    \fill[fill opacity=.5, \relleno] (6.25,3) rectangle (6.5,2);
    \draw[thin, dashed] (6.25,2) -- (6.25,3);
    \draw[thin, dashed] (6.5,2) -- (6.5,3);
    \draw[very thick, \colfun] (6,2) -- (6.25,2);
    \draw[very thick, \colfun] (6.25,3) -- (6.5,3);
    \draw[very thick, \colfun] (6.5,2) -- (7,2);

    % Fila 2
    \draw[->] (-0.2,0) -- (1.2,0);
    \draw[->] (0,-0.2) -- (0,1.2);
    \fill[fill opacity=.5, \relleno] (0.5,1) rectangle (0.75,0);
    \draw[thin, dashed] (0.5,0) -- (0.5,1);
    \draw[thin, dashed] (0.75,0) -- (0.75,1);
    \draw[very thick, \colfun] (0,0) -- (0.5,0);
    \draw[very thick, \colfun] (0.5,1) -- (0.75,1);
    \draw[very thick, \colfun] (0.75,0) -- (1,0);
    
    \draw[->] (1.8,0) -- (3.2,0);
    \draw[->] (2,-0.2) -- (2,1.2);
    \fill[fill opacity=.5, \relleno] (2.75,1) rectangle (3,0);
    \draw[thin, dashed] (2.75,0) -- (2.75,1);
    \draw[thin, dashed] (3,0) -- (3,1);
    \draw[very thick, \colfun] (2,0) -- (2.75,0);
    \draw[very thick, \colfun] (2.75,1) -- (3,1);

    \draw[->] (3.8,0) -- (5.2,0);
    \draw[->] (4,-0.2) -- (4,1.2);
    \fill[fill opacity=.5, \relleno] (4,1) rectangle (4.125,0);
    \draw[thin, dashed] (4.125,0) -- (4.125,1);
    \draw[very thick, \colfun] (4,1) -- (4.125,1);
    \draw[very thick, \colfun] (4.125,0) -- (5,0);
    
    \draw[->] (5.8,0) -- (7.2,0);
    \draw[->] (6,-0.2) -- (6,1.2);
    \fill[fill opacity=.5, \relleno] (6.125,1) rectangle (6.25,0);
    \draw[thin, dashed] (6.125,0) -- (6.125,1);
    \draw[thin, dashed] (6.25,0) -- (6.25,1);
    \draw[very thick, \colfun] (6,0) -- (6.125,0);
    \draw[very thick, \colfun] (6.125,1) -- (6.25,1);
    \draw[very thick, \colfun] (6.25,0) -- (7,0);
  \end{tikzpicture}
  \caption{Primers termes de la successió de funcions $(f_n)_n$.}
  \label{fig:exemple-no-conver}
\end{figure}

Aquest últim exemple motiva la definició de \textbf{límit
  inferior}. Hem vist que no sempre podrem intercanviar la integral
amb el límit. No obstant, encara que una successió no siga convergent,
el seu límit inferior sempre existirà i es pot demostrar el Lema de Fatou (Teorema \ref{teo:fatou}).

\begin{defi}[Límit inferior]
  Siga $(x_n)_n$ una successió de nombres reals, aleshores el límit
  inferior de la successió $(x_n)_n$ es defineix com
  \[
    \liminf_{n \to \infty} x_n = \lim_{n \to \infty} \inf_{k \geq n} x_n.
  \]
\end{defi}

\begin{teorema}[Lema de Fatou]\label{teo:fatou}
  Siga $(f_n)_n$ una successió de funcions mesurables i positives en
  $L^p(\Omega)$, aleshores
  \[
    \int_{\Omega} \liminf_{n \to \infty} f_n(x) \, \mathrm{d}x \leq
    \liminf_{n \to \infty} \int_{\Omega} f_n(x) \, \mathrm{d}x.
  \]
\end{teorema}

\begin{teorema}[$L^p(\Omega)$ és un espai de Banach]
  Per a $1 \leq p < \infty$, l'espai de Lebesgue $L^p(\Omega)$ és un
  espai de Banach.
\end{teorema}

\begin{coro}
  Per a $1 \leq p < \infty$, si $f_n \to f$ en $L^p(\Omega)$,
  aleshores existeix una subsuccessió $(f_{n_k})_k$ de $(f_n)_n$ tal
  que $f_{n_k} \to f$ quasi per totes parts en $\Omega$.
\end{coro}

\begin{teorema}
  Per a $1 \leq p < q < \infty$, es compleix que
  $L^q(\Omega) \subset L^p(\Omega)$ amb inclusió contínua. És a dir,
  la convergència d'una successió en $L^q(\Omega)$ implica la
  convergència d'aquesta en $L^p(\Omega)$.
\end{teorema}

\section{Operadors entre espais normats}

\begin{defi}[Norma d'un operador fitat]
  Siguen $E$ i $F$ dos espais normats i $T : E \to F$ una aplicació
  lineal de $E$ en $F$. Diem que $T$ és un operador fitat si existeix
  una constant $C > 0$ tal que per a tot $x \in E$ es compleix que
  \[
    \|T(x)\|_F \leq C, \quad \forall\, x \in E, \;\; \|x\|_E \leq 1.
  \]
  La norma d'un operador fitat $T$ es defineix com
  \[
    \|T\| = \sup_{\|x\|_E \leq 1} \|T(x)\|_F.
  \]
\end{defi}

\begin{nota}
  La norma de l'operador $T$ és la menor constant $C > 0$ que verifica
  la desigualtat anterior, és a dir,
  \begin{align*}
    \|T\| &= \inf\{C > 0 : \|T(x)\|_F \leq C, \;\; \forall\, x \in E, \;\; \|x\|_E \leq 1\} \\
          &= \inf\{C > 0 : \|T(x)\|_F \leq C \|x\|_E, \;\; \forall\, x \in E\}.
  \end{align*}
  on la segon igualtat se segueix de l'homogeneïtat de la norma i la
  linealitat de $T$.
\end{nota}

\begin{exemple}
  Fixat $a \in \mathbb{R}^n$ i $1 \leq p \leq +\infty$, considerem
  l'aplicació lineal
  \begin{equation*}
    \begin{array}{lrcl}
      \varphi : & (\mathbb{R}^n, \|\cdot\|_p) &\longrightarrow& \mathbb{R} \\
      & x &\longmapsto& \langle a, x \rangle = \sum\limits_{i=1}^{n}a_ix_i.
    \end{array}
  \end{equation*}
  Aleshores, si $q$ és el conjugat de $p$, la norma de $\varphi$ és
  $\|\varphi\| = \|a\|_q$.
\end{exemple}

\begin{exemple}
  L'operador
  \[
    \begin{array}{lrcl}
      T : & (\mathcal{C}[0,1], \|\cdot\|_\infty) &\longrightarrow& (\mathcal{C}[0,1], \|\cdot\|_\infty), \\
      & f &\longmapsto& T(f) = \int_0^x f(t)\, \mathrm{d}t, \quad x \in [0,1],
    \end{array}
  \]
  és un operador lineal de norma $1$.
\end{exemple}

\begin{defi}[L'espai dels operadors fitats]
  Donats dos espais normats $E$ i $F$, denotem per
  $(L(E,F), \|\cdot\|)$ al conjunt dels operadors fitats d'$E$ en $F$
  amb la norma dels operadors fitats. Aquest conjunt és un espai
  vectorial sobre el cos $\mathbb{K}$ amb les operacions de suma i
  producte per escalars definides de la següent manera,
  \begin{align*}
    (T_1 + T_2)(x) &= T_1(x) + T_2(x), \\
    (\lambda T)(x) &= \lambda T(x),
  \end{align*}
  per a tot $T_1,\, T_2 \in L(E, F, \|\cdot\|)$,
  $\lambda \in \mathbb{K}$ i $x \in E$.
\end{defi}

\begin{teorema}
  L'espai $(L(E,F), \|\cdot\|)$ és un espai vectorial normat amb la
  norma dels operadors fitats. A més, si $E$, $F$ i $G$ són espais
  normats i $T \in (L(E,F), \|\cdot\|)$ i $S \in (L(F,G), \|\cdot\|)$,
  aleshores $S \circ T \in (L(E,G), \|\cdot\|)$ i
  $\|S \circ T\| \leq \|S\| \|T\|$.
\end{teorema}

\begin{teorema}
  Si $F$ és un espai de Banach, aleshores l'espai
  $(L(E,F), \|\cdot\|)$ és un espai de Banach.
\end{teorema}

\begin{defi}[Espai dual]
  Donat un espai normat $E$, l'espai dual de $E$, denotat per $E^*$,
  és l'espai dels funcionals lineals i continus de $E$ en
  $\mathbb{K}$.
\end{defi}

\begin{coro}
  Si $E$ és un espai de Banach, aleshores $E^*$ és un espai de Banach.
\end{coro}

\end{document}

%%% Local Variables:
%%% mode: LaTeX
%%% TeX-master: t
%%% End:
